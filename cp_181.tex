\hypertarget{engineering-thermodynamics-lab-1-pvt}{%
\section{\texorpdfstring{\textbf{\textbf{Engineering Thermodynamics: Lab
1:
PVT}}}{Engineering Thermodynamics: Lab 1: PVT}}\label{engineering-thermodynamics-lab-1-pvt}}

\hypertarget{juliet-nwagwu-ume-ezeoke---friday-september-9-2019-lab-at-900am}{%
\subparagraph{Juliet Nwagwu Ume-Ezeoke - Friday, September 9, 2019 Lab
at
9:00am}\label{juliet-nwagwu-ume-ezeoke---friday-september-9-2019-lab-at-900am}}

\hypertarget{i.-introduction}{%
\section{I. Introduction}\label{i.-introduction}}

The purpose of this lab was to examine the relationship between the
properties, phases, and states of a pure substance. We used sulfur
hexaflouride (SF6), a heavy gas for the experiment. By adjusting the
volume that the gas was in, were able to adjust the pressure, and
eventually, the state.

\hypertarget{ii.-procedure}{%
\section{II. Procedure}\label{ii.-procedure}}

\begin{enumerate}
\def\labelenumi{\arabic{enumi}.}
\tightlist
\item
  We set up an excel spreadsheet to record our data.
\item
  We designated everyone tasks, and then ensured that the temperature
  reading was 30ºC at the maximum volume.
\item
  We gathered the tabulated data from NIST for the 20ºC - 50ºC isotherms
  (at inervals of 10ºC).
\item
  We decreased the volume by rotating the handwheel until we reached
  about 10mm on the fixed scale recorded the corresponding pressure,
  first stopping at 13mm to allow the material to reach equilibrium.
\item
  At this point, we allowed the pressure to hit 4MPa. We waited for five
  minutes before recording this absolute pressure.
\item
  We then increased and decreased the volumes according to given
  intervals, recording the corresponding pressures.
\item
  We analyzed our data and ensured that the pressure-volume relationship
  was followed that of a standard P-V curve.
\end{enumerate}

\hypertarget{equipment}{%
\subsection{Equipment}\label{equipment}}

\hypertarget{iii.-data}{%
\section{III. Data}\label{iii.-data}}

\begin{Shaded}
\begin{Highlighting}[]
\ImportTok{import}\NormalTok{ numpy }\ImportTok{as}\NormalTok{ np}
\ImportTok{import}\NormalTok{ pandas }\ImportTok{as}\NormalTok{ pd}
\ImportTok{import}\NormalTok{ matplotlib.pyplot }\ImportTok{as}\NormalTok{ plt}
\ImportTok{import}\NormalTok{ math}
\ImportTok{from}\NormalTok{ IPython.display }\ImportTok{import}\NormalTok{ Math}
\end{Highlighting}
\end{Shaded}

\begin{Shaded}
\begin{Highlighting}[]
\NormalTok{data }\OperatorTok{=}\NormalTok{ pd.read_csv(}\StringTok{'lab1_data.csv'}\NormalTok{, skiprows}\OperatorTok{=}\DecValTok{3}\NormalTok{, header}\OperatorTok{=}\VariableTok{None}\NormalTok{)}
\end{Highlighting}
\end{Shaded}

\begin{Shaded}
\begin{Highlighting}[]
\NormalTok{df }\OperatorTok{=}\NormalTok{ pd.DataFrame(data)}
\NormalTok{df.columns }\OperatorTok{=}\NormalTok{ [}\StringTok{'e30_p'}\NormalTok{, }\StringTok{'e30_v'}\NormalTok{, }\StringTok{'20_p'}\NormalTok{, }\StringTok{'20_v'}\NormalTok{, }\StringTok{'30_p'}\NormalTok{, }\StringTok{'30_v'}\NormalTok{, }\StringTok{'40_p'}\NormalTok{, }\StringTok{'40_v'}\NormalTok{, }\StringTok{'50_p'}\NormalTok{, }\StringTok{'50_v'}\NormalTok{]}
\NormalTok{df.head() }\CommentTok{# see all experimental data in appendix}
\CommentTok{# note that the experimental volume e30_v != specific volume, while the others are }
\CommentTok{# make this look nice -- divide the volume by the # of grams }
\CommentTok{# put experimental data in a nice thing }
\end{Highlighting}
\end{Shaded}

e30\_p

e30\_v

20\_p

20\_v

30\_p

30\_v

40\_p

40\_v

50\_p

50\_v

0

1.00

14.444

0.091667

180.070

0.091667

186.430

0.091667

192.760

0.091667

199.090

1

1.16

12.560

0.183330

89.032

0.183330

92.282

0.183330

95.518

0.183330

98.741

2

1.33

10.676

0.275000

58.672

0.275000

60.891

0.275000

63.095

0.275000

65.286

3

1.59

8.792

0.366670

43.482

0.366670

45.188

0.366670

46.878

0.366670

48.554

4

1.90

6.908

0.458330

34.360

0.458330

35.760

0.458330

37.143

0.458330

38.511

\begin{Shaded}
\begin{Highlighting}[]
\CommentTok{# graph of measured data }
\NormalTok{plt.plot(df[}\StringTok{'e30_v'}\NormalTok{].dropna(), df[}\StringTok{'e30_p'}\NormalTok{].dropna(), }\StringTok{'ro'}\NormalTok{, markersize}\OperatorTok{=}\DecValTok{3}\NormalTok{)}
\NormalTok{plt.xlabel(}\StringTok{'Volume (cm^3)'}\NormalTok{)}
\NormalTok{plt.ylabel(}\StringTok{'Pressure (MPa)'}\NormalTok{)}
\NormalTok{plt.title(}\StringTok{'Measured Experimental Data: SF6 at 30ºC')}
\end{Highlighting}
\end{Shaded}

Text(0.5, 1.0, `Measured Experimental Data: SF6 at 30ºC')

\begin{figure}
\centering
\includegraphics{cp_181_lab_1_files/cp_181_lab_1_5_1.png}
\caption{png}
\end{figure}

\begin{Shaded}
\begin{Highlighting}[]
\CommentTok{# create a new dataframe with units of pascals and m^3 for calculations only}
\NormalTok{df2 }\OperatorTok{=}\NormalTok{ df.copy()}

\CommentTok{# transform the data to have pressure be in regular pascals ( multiply by 10^6)}
\NormalTok{df2[df2.columns[::}\DecValTok{2}\NormalTok{]] }\OperatorTok{=}\NormalTok{ df[df.columns[::}\DecValTok{2}\NormalTok{]] }\OperatorTok{*}\NormalTok{ math.}\BuiltInTok{pow}\NormalTok{(}\DecValTok{10}\NormalTok{,}\DecValTok{6}\NormalTok{)}


\CommentTok{# change volume to be in units of m^3 insead of cm^3}
\NormalTok{df2[df2.columns[}\DecValTok{1}\NormalTok{::}\DecValTok{2}\NormalTok{]] }\OperatorTok{=}\NormalTok{ df[df.columns[}\DecValTok{1}\NormalTok{::}\DecValTok{2}\NormalTok{]] }\OperatorTok{/}\NormalTok{ (math.}\BuiltInTok{pow}\NormalTok{(}\DecValTok{10}\NormalTok{,}\DecValTok{6}\NormalTok{)) }
\end{Highlighting}
\end{Shaded}

\hypertarget{iv.-analysis}{%
\section{IV. Analysis}\label{iv.-analysis}}

\hypertarget{determine-the-mass-in-grams-and-the-number-of-moles-of-sf6-in-the-apparatus-by-using-the-experimental-measurement-that-should-most-nearly-reflect-ideal-gas-behavior.-do-the-same-for-the-nist-data.}{%
\paragraph{1. Determine the mass (in grams) and the number of moles of
SF6 in the apparatus by using the experimental measurement that should
most nearly reflect ideal gas behavior. Do the same for the NIST
data.}\label{determine-the-mass-in-grams-and-the-number-of-moles-of-sf6-in-the-apparatus-by-using-the-experimental-measurement-that-should-most-nearly-reflect-ideal-gas-behavior.-do-the-same-for-the-nist-data.}}

We can begin by rearranging the ideal gas law to give the number of
moles based on the known values of pressure, volume, temperature, and
the universal gas constant. The mass of gas in apparatus, M\_g will be
the product of the number of moles in tha apparatus n, and the molar
mass of of SF6, mm, which is 146.06 g/mol.\\
The experimental measurement that is most like an ideal gas occurs when
the volume is the greatest at 14.4 cm\^{}3. For the NIST data, the
tabulated meausrement with the greatest volume will also be used.

\begin{Shaded}
\begin{Highlighting}[]
\CommentTok{%%latex}
\KeywordTok{\textbackslash{}begin}\NormalTok{\{}\ExtensionTok{align}\NormalTok{\}}
\SpecialStringTok{PV = nRT }\SpecialCharTok{\textbackslash{}\textbackslash{}}
\SpecialStringTok{n = }\SpecialCharTok{\textbackslash{}frac}\SpecialStringTok{\{PV\}\{RT\} }\SpecialCharTok{\textbackslash{}\textbackslash{}}
\SpecialStringTok{M_g = n * mm}

\KeywordTok{\textbackslash{}end}\NormalTok{\{}\ExtensionTok{align}\NormalTok{\}}
\end{Highlighting}
\end{Shaded}

\begin{align}
PV = nRT \\
n = \frac{PV}{RT} \\
M_g = n * mm

\end{align}

\begin{Shaded}
\begin{Highlighting}[]
\NormalTok{R }\OperatorTok{=} \FloatTok{8.314} \CommentTok{# universal gas constant ----  J/K-mole == N_a * k == (# of molecules * joules)/(K *moles)}
\NormalTok{mm }\OperatorTok{=} \FloatTok{146.06} \CommentTok{# molor mass of SF6 g/mol  }


\NormalTok{vol }\OperatorTok{=} \BuiltInTok{list}\NormalTok{(df.columns)[}\DecValTok{1}\NormalTok{::}\DecValTok{2}\NormalTok{]}
\NormalTok{pres }\OperatorTok{=} \BuiltInTok{list}\NormalTok{(df.columns)[::}\DecValTok{2}\NormalTok{]}
\NormalTok{Ts }\OperatorTok{=}\NormalTok{ [}\DecValTok{30}\NormalTok{, }\DecValTok{20}\NormalTok{, }\DecValTok{30}\NormalTok{, }\DecValTok{40}\NormalTok{, }\DecValTok{50}\NormalTok{]}
\NormalTok{ns }\OperatorTok{=}\NormalTok{ []}
\NormalTok{m_gs }\OperatorTok{=}\NormalTok{ []}
\ControlFlowTok{for}\NormalTok{ i }\KeywordTok{in} \BuiltInTok{range}\NormalTok{(}\DecValTok{0}\NormalTok{, }\BuiltInTok{len}\NormalTok{(vol)):}
\NormalTok{    T }\OperatorTok{=}\NormalTok{ Ts[i] }\OperatorTok{+} \FloatTok{273.15} \CommentTok{# Kelvin from Celcius for the experiment }
\NormalTok{    RT }\OperatorTok{=}\NormalTok{ R}\OperatorTok{*}\NormalTok{T}
    
\NormalTok{    nss }\OperatorTok{=}\NormalTok{ df2.loc[}\DecValTok{0}\NormalTok{, pres[i]] }\OperatorTok{*}\NormalTok{ df2.loc[}\DecValTok{0}\NormalTok{, vol[i]] }\OperatorTok{/}\NormalTok{ RT }\CommentTok{# number of moles}
\NormalTok{    ns.append(nss)}
\NormalTok{    m_gs.append(nss }\OperatorTok{*}\NormalTok{ mm) }\CommentTok{# mass in grams }

\NormalTok{data }\OperatorTok{=}\NormalTok{ \{}\StringTok{'Number of moles'}\NormalTok{:ns, }\StringTok{'Mass in grams'}\NormalTok{: m_gs\}}
\NormalTok{temps }\OperatorTok{=}\NormalTok{ [}\StringTok{'exp 30ºC','}\NormalTok{20ºC', }\StringTok{'30ºC', '}\NormalTok{40ºC', }\StringTok{'50ºC']}
\StringTok{df3 = pd.DataFrame(data, index = temps) }
\StringTok{df3              }
\end{Highlighting}
\end{Shaded}

Number of moles

Mass in grams

exp 30ºC

0.005731

0.837050

20ºC

0.006773

0.989204

30ºC

0.006780

0.990359

40ºC

0.006787

0.991286

50ºC

0.006793

0.992155

\hypertarget{plot-p-v-diagrams-of-your-data-and-nist-data-in-one-graph.-for-your-isotherms-connect-points-to-get-a-smooth-curve.}\NormalTok{matplotlib inline}
\KeywordTok{def}\NormalTok{ g_plot():}
\NormalTok{    plt.xlabel(}\StringTok{'Volume (cm^3)'}\NormalTok{)}
\NormalTok{    plt.ylabel(}\StringTok{'Pressure (MPa)'}\NormalTok{)}
\NormalTok{    plt.title(}\StringTok{'Measured Experimental Data and NIST: SF6 at 30ºC')}
\StringTok{    lines = plt.plot(df['}\NormalTok{e30_v}\StringTok{'], df['}\NormalTok{e30_p}\StringTok{'], '}\NormalTok{mo}\StringTok{',}
\StringTok{            df['}\DecValTok{20}\NormalTok{_v}\StringTok{'].dropna(), df['}\DecValTok{20}\NormalTok{_p}\StringTok{'].dropna(), '}\VerbatimStringTok{r',}
\VerbatimStringTok{            df['}\DecValTok{30}\NormalTok{_v}\StringTok{'], df['}\DecValTok{30}\NormalTok{_p}\StringTok{'], '}\NormalTok{b}\StringTok{',}
\StringTok{            df['}\DecValTok{40}\NormalTok{_v}\StringTok{'], df['}\DecValTok{40}\NormalTok{_p}\StringTok{'], '}\NormalTok{g}\StringTok{',}
\StringTok{            df['}\DecValTok{50}\NormalTok{_v}\StringTok{'], df['}\DecValTok{50}\NormalTok{_p}\StringTok{'], '}\NormalTok{y}\StringTok{',}
\StringTok{             markersize=3)}
\StringTok{    plt.legend((temps))}

\StringTok{g_plot()}
\StringTok{plt.axis([0, 15 , 0.4, 6]) # zoomed in version, see complete graph in appendix}
\StringTok{# experimental data is intentionally left dotted to differentiate}
\end{Highlighting}
\end{Shaded}

{[}0, 15, 0.4, 6{]}

\begin{figure}
\centering
\includegraphics{cp_181_lab_1_files/cp_181_lab_1_12_1.png}
\caption{png}
\end{figure}

\hypertarget{discuss-the-state-of-sf6-during-the-experiments.}{%
\paragraph{3. Discuss the state of SF6 during the
experiments.}\label{discuss-the-state-of-sf6-during-the-experiments.}}

State can be described by two properties of a pure substance. Here we
have three properties that are known to us over the time period of the
experiment, pressure, temperature and volume. We want the temperature to
be constant at 30ºC for the duration of the experiment. Therefore, we
allowed additional time for the temperature to equilibriate before
adjusting the setup. We adjusted the volume of the space occupied by the
SF6 and by cranking the handwheel. Decreasing the volume like this
allowed us to observe the increase in pressure. We were able to somewhat
observe the change in phase from gas to liquid as the volume adjusted.
\#\#\#\# 4. Discuss the difference between isotherms set. How
higher/lower temperature affects the curve?\\
Having a higher temperature usually resulted in having a ``quicker''
transition. This means that there were far fewer volumes for which the
SF6 was in a two phase states when the temperature was higher. The 20ºC
isotherm has the greatest change in volume occur during the phase
transition from gas to liquid. For the 50ºC isotherm it is almost non
existent. \#\#\#\# 5. Determine the mass (in grams) of each liquid SF6
and gas SF6 of each data point of measured data.

From question number one, we know how to simply calculate the mass with
the know values.

\begin{Shaded}
\begin{Highlighting}[]
\CommentTok{%%latex}
\KeywordTok{\textbackslash{}begin}\NormalTok{\{}\ExtensionTok{align}\NormalTok{\}}
\SpecialStringTok{M_g = }\SpecialCharTok{\textbackslash{}frac}\SpecialStringTok{\{PV*mm\}\{RT\}}
\KeywordTok{\textbackslash{}end}\NormalTok{\{}\ExtensionTok{align}\NormalTok{\}}
\NormalTok{# multiply by mass to get m^3/kg -- total mass - from the top -- ration of liquid vs solid -- pvnrt --}
\NormalTok{# to get # of moles, subtract from mass}
\end{Highlighting}
\end{Shaded}

\begin{align}
M_g = \frac{PV*mm}{RT}
\end{align} \# multiply by mass to get m\^{}3/kg -- total mass - from
the top -- ration of liquid vs solid -- pvnrt -- \# to get \# of moles,
subtract from mass

\begin{Shaded}
\begin{Highlighting}[]
\NormalTok{T_ }\OperatorTok{=} \DecValTok{30} \OperatorTok{+} \FloatTok{273.15}
\NormalTok{RT_ }\OperatorTok{=}\NormalTok{ R }\OperatorTok{*}\NormalTok{ T_}
\NormalTok{m_gas_og }\OperatorTok{=}\NormalTok{ m_gs[}\DecValTok{0}\NormalTok{] }\CommentTok{# the greatest mass of gas that was ever present }
\NormalTok{m_gas }\OperatorTok{=}\NormalTok{ df2[}\StringTok{'e30_p'}\NormalTok{]}\OperatorTok{*}\NormalTok{df2[}\StringTok{'e30_v'}\NormalTok{]}\OperatorTok{*}\NormalTok{mm}\OperatorTok{/}\NormalTok{RT_ }\CommentTok{# current amount of gas from the ideal gas law }
\NormalTok{m_liq }\OperatorTok{=}\NormalTok{ m_gas_og }\OperatorTok{-}\NormalTok{ m_gas}
\CommentTok{# print(m_gas.dropna())}
\CommentTok{# print(m_liq.dropna())}
\NormalTok{masses }\OperatorTok{=}\NormalTok{ \{}\StringTok{'Mass of Gas (g)'}\NormalTok{: m_gas.dropna(), }\StringTok{'Mass of Liquid (g)'}\NormalTok{: m_liq.dropna()\}}
\NormalTok{df4 }\OperatorTok{=}\NormalTok{ pd.DataFrame(masses) }
\NormalTok{df4}
\end{Highlighting}
\end{Shaded}

Mass of Gas (g)

Mass of Liquid (g)

0

0.837050

0.000000

1

0.844328

-0.007279

2

0.822856

0.014193

3

0.810118

0.026931

4

0.760623

0.076426

5

0.675463

0.161587

6

0.556820

0.280230

7

0.508417

0.328633

8

0.462197

0.374853

9

0.167410

0.669640

10

0.144309

0.692740

11

0.129925

0.707125

12

0.152525

0.684525

13

0.175635

0.661415

14

0.221855

0.615195

15

0.267019

0.570031

16

0.311819

0.525230

17

0.357675

0.479375

18

0.403531

0.433519

\hypertarget{comment-on-the-accuracy-and-agreement-between-the-labcourse-data-and-tabulated-data-from-nist.-discuss-possible-sources-of-disagreement.-why-might-the-experimental-data-disagree-with-the-tabulated-data}{%
\paragraph{6. Comment on the accuracy and agreement between the
lab/course data and tabulated data from NIST. Discuss possible sources
of disagreement. Why might the experimental data disagree with the
tabulated
data?}\label{comment-on-the-accuracy-and-agreement-between-the-labcourse-data-and-tabulated-data-from-nist.-discuss-possible-sources-of-disagreement.-why-might-the-experimental-data-disagree-with-the-tabulated-data}}

From observing the graph, we can see that the 30ºC isotherm that we
recorded during the experiment generally follows that of the NIST data.
Important differences can be observed during the phase trasnsition,
where the experimental pressures are about 100kPa below the NIST
pressures. Also as the volume increases past 9 cm\^{}3, we see that the
experimental pressures are again lower than those of the tabulated data.
This might be the case because we did not allowe the substance to reach
complete equilibrium before capturing our data. While we did often wait
the prescribed time, in addition to checking the meter was no longer
fluctuating, there might have been some disconnect.

\hypertarget{v.-conclusions}{%
\section{V. Conclusions}\label{v.-conclusions}}

In conclusion, we found the expected relationship between the pressure
and volume for the SF6 to be as expected. While the volume was large,
and the pressure low, the substance remained in its gaseous state. As
the volume decreased, and the pressure inside the container increased,
the substance entered a two-phase state of liquid and gas. It remained
here for a large number of changes in volume, as the pressure flatlined
during the two-phase state. Once the phase transition had occured, the
pressure increased again rapidly.

\hypertarget{vi.-appendeix}{%
\section{VI. Appendeix}\label{vi.-appendeix}}

\hypertarget{a.-complete-graph-of-nist-and-measured-experimental-data}{%
\subparagraph{A. Complete graph of NIST and measured experimental
data}\label{a.-complete-graph-of-nist-and-measured-experimental-data}}

\begin{Shaded}
\begin{Highlighting}[]
\NormalTok{g_plot()}
\end{Highlighting}
\end{Shaded}

\begin{figure}
\centering
\includegraphics{cp_181_lab_1_files/cp_181_lab_1_20_0.png}
\caption{png}
\end{figure}

\hypertarget{b.-all-experimental-data}{%
\subparagraph{B. All experimental data}\label{b.-all-experimental-data}}

\begin{Shaded}
\begin{Highlighting}[]
\NormalTok{df}
\end{Highlighting}
\end{Shaded}

e30\_p

e30\_v

20\_p

20\_v

30\_p

30\_v

40\_p

40\_v

50\_p

50\_v

0

1.00

14.4440

0.091667

180.07000

0.091667

186.43000

0.091667

192.76000

0.091667

199.09000

1

1.16

12.5600

0.183330

89.03200

0.183330

92.28200

0.183330

95.51800

0.183330

98.74100

2

1.33

10.6760

0.275000

58.67200

0.275000

60.89100

0.275000

63.09500

0.275000

65.28600

3

1.59

8.7920

0.366670

43.48200

0.366670

45.18800

0.366670

46.87800

0.366670

48.55400

4

1.90

6.9080

0.458330

34.36000

0.458330

35.76000

0.458330

37.14300

0.458330

38.51100

5

2.32

5.0240

0.550000

28.27100

0.550000

29.46900

0.550000

30.64800

0.550000

31.81300

6

2.55

3.7680

0.641670

23.91500

0.641670

24.97000

0.641670

26.00600

0.641670

27.02500

7

2.54

3.4540

0.733330

20.64100

0.733330

21.59100

0.733330

22.52000

0.733330

23.43200

8

2.54

3.1400

0.825000

18.08900

0.825000

18.95900

0.825000

19.80600

0.825000

20.63500

9

4.00

0.7222

0.916670

16.04100

0.916670

16.84900

0.916670

17.63200

0.916670

18.39500

10

3.11

0.8007

1.008300

14.36000

1.008300

15.11800

1.008300

15.85000

1.008300

16.56000

11

2.55

0.8792

1.100000

12.95200

1.100000

13.67200

1.100000

14.36200

1.100000

15.02900

12

2.54

1.0362

1.191700

11.75500

1.191700

12.44400

1.191700

13.10000

1.191700

13.73100

13

2.54

1.1932

1.283300

10.72300

1.283300

11.38700

1.283300

12.01500

1.283300

12.61700

14

2.54

1.5072

1.375000

9.82150

1.375000

10.46700

1.375000

11.07200

1.375000

11.64900

15

2.53

1.8212

1.466700

9.02580

1.466700

9.65770

1.466700

10.24500

1.466700

10.80100

16

2.52

2.1352

1.558300

8.31620

1.558300

8.93910

1.558300

9.51160

1.558300

10.05000

17

2.52

2.4492

1.650000

7.67710

1.650000

8.29570

1.650000

8.85700

1.650000

9.38080

18

2.52

2.7632

1.741700

7.09600

1.741700

7.71510

1.741700

8.26830

1.741700

8.78010

19

NaN

NaN

1.833300

6.56260

1.833300

7.18730

1.833300

7.73540

1.833300

8.23760

20

NaN

NaN

1.925000

6.06760

1.925000

6.70410

1.925000

7.25000

1.925000

7.74470

21

NaN

NaN

2.016700

5.60260

2.016700

6.25860

2.016700

6.80550

2.016700

7.29470

22

NaN

NaN

2.099800

5.19930

2.108300

5.84490

2.108300

6.39600

2.108300

6.88170

23

NaN

NaN

2.099800

0.71723

2.200000

5.45790

2.200000

6.01700

2.200000

6.50090

24

NaN

NaN

2.108300

0.71707

2.291700

5.09260

2.291700

5.66420

2.291700

6.14850

25

NaN

NaN

2.200000

0.71541

2.383300

4.74450

2.383300

5.33420

2.383300

5.82090

26

NaN

NaN

2.291700

0.71380

2.475000

4.40850

2.475000

5.02380

2.475000

5.51510

27

NaN

NaN

2.383300

0.71224

2.566700

4.07850

2.566700

4.73030

2.566700

5.22870

28

NaN

NaN

2.475000

0.71073

2.655500

3.75540

2.658300

4.45100

2.658300

4.95930

29

NaN

NaN

2.566700

0.70926

2.655500

0.78032

2.750000

4.18320

2.750000

4.70500

\ldots{}

\ldots{}

\ldots{}

\ldots{}

\ldots{}

\ldots{}

\ldots{}

\ldots{}

\ldots{}

\ldots{}

\ldots{}

572

NaN

NaN

52.342000

0.55413

52.342000

0.56342

52.342000

0.57295

52.342000

0.58275

573

NaN

NaN

52.433000

0.55403

52.433000

0.56330

52.433000

0.57282

52.433000

0.58261

574

NaN

NaN

52.525000

0.55392

52.525000

0.56319

52.525000

0.57269

52.525000

0.58247

575

NaN

NaN

52.617000

0.55381

52.617000

0.56307

52.617000

0.57256

52.617000

0.58232

576

NaN

NaN

52.708000

0.55370

52.708000

0.56295

52.708000

0.57243

52.708000

0.58218

577

NaN

NaN

52.800000

0.55360

52.800000

0.56283

52.800000

0.57230

52.800000

0.58204

578

NaN

NaN

52.892000

0.55349

52.892000

0.56271

52.892000

0.57218

52.892000

0.58190

579

NaN

NaN

52.983000

0.55338

52.983000

0.56260

52.983000

0.57205

52.983000

0.58176

580

NaN

NaN

53.075000

0.55328

53.075000

0.56248

53.075000

0.57192

53.075000

0.58162

581

NaN

NaN

53.167000

0.55317

53.167000

0.56236

53.167000

0.57179

53.167000

0.58148

582

NaN

NaN

53.258000

0.55307

53.258000

0.56225

53.258000

0.57166

53.258000

0.58134

583

NaN

NaN

53.350000

0.55296

53.350000

0.56213

53.350000

0.57153

53.350000

0.58120

584

NaN

NaN

53.442000

0.55285

53.442000

0.56201

53.442000

0.57141

53.442000

0.58106

585

NaN

NaN

53.533000

0.55275

53.533000

0.56190

53.533000

0.57128

53.533000

0.58092

586

NaN

NaN

53.625000

0.55264

53.625000

0.56178

53.625000

0.57115

53.625000

0.58078

587

NaN

NaN

53.717000

0.55254

53.717000

0.56167

53.717000

0.57102

53.717000

0.58064

588

NaN

NaN

53.808000

0.55243

53.808000

0.56155

53.808000

0.57090

53.808000

0.58050

589

NaN

NaN

53.900000

0.55233

53.900000

0.56144

53.900000

0.57077

53.900000

0.58036

590

NaN

NaN

53.992000

0.55222

53.992000

0.56132

53.992000

0.57065

53.992000

0.58022

591

NaN

NaN

54.083000

0.55212

54.083000

0.56121

54.083000

0.57052

54.083000

0.58009

592

NaN

NaN

54.175000

0.55202

54.175000

0.56109

54.175000

0.57039

54.175000

0.57995

593

NaN

NaN

54.267000

0.55191

54.267000

0.56098

54.267000

0.57027

54.267000

0.57981

594

NaN

NaN

54.358000

0.55181

54.358000

0.56086

54.358000

0.57014

54.358000

0.57967

595

NaN

NaN

54.450000

0.55170

54.450000

0.56075

54.450000

0.57002

54.450000

0.57953

596

NaN

NaN

54.542000

0.55160

54.542000

0.56063

54.542000

0.56989

54.542000

0.57940

597

NaN

NaN

54.633000

0.55150

54.633000

0.56052

54.633000

0.56977

54.633000

0.57926

598

NaN

NaN

54.725000

0.55139

54.725000

0.56041

54.725000

0.56964

54.725000

0.57912

599

NaN

NaN

54.817000

0.55129

54.817000

0.56029

54.817000

0.56952

54.817000

0.57899

600

NaN

NaN

54.908000

0.55119

54.908000

0.56018

54.908000

0.56939

54.908000

0.57885

601

NaN

NaN

55.000000

0.55108

55.000000

0.56007

55.000000

0.56927

55.000000

0.57872

602 rows × 10 columns

\begin{Shaded}
\begin{Highlighting}[]

\end{Highlighting}
\end{Shaded}

